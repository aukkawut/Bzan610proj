\documentclass[12pt]{article}
\usepackage{booktabs}
\usepackage{amssymb,amsmath, amsthm, endnotes,setspace,enumerate,ulem,color,xspace,lscape}
\usepackage{amsthm,subfigure,graphics,epsfig, hyperref}
\usepackage[sort]{natbib}
\allowdisplaybreaks[1]
\normalem
\newcommand{\textR}[1]{\textcolor{blue}{\texttt{#1}}}
\newcommand{\R}{\textR{R}}
\newtheorem{post}{Postulate}
\newtheorem{theorem}{Theorem}
\newtheorem{definition}{Definition}
\begin{document}

\title{Adversarial Pheromone-based Swarm Robots Dynamics as System of Stochastic Partial Differential Equations: Continuous-time Information Design Perspective}  

\author{
Aukkawut Ammartayakun   \thanks{Bredesen Center for Interdisciplinary Research and Graduate Education, University of Tennessee, Knoxville.  \\Email: {\tt aammartayakun@tennessee.edu}}  
}

\maketitle

\begin{center}
\textbf{Abstract}
\end{center}

In the foraging task, the goal for agents are to reach the objective and to accumulate said objective by bringing those back to the start. However, as shown by Aswale et al. (2022), if the information is communicate through the pheromone, then there exists the adversarial strategy that can sabotage the task. In this work, the theoretical framework for the adversarial foraging task will be presented along with the insights from the information theory perspective. System of stochastic partial differential equations is used to formulate the framework along with mean-field theory. 

\noindent
\textbf{Keywords}: Byzantine Model, Foraging Task, Swarm Robotics, Adversarial Behavior, Pheromone-based Robots

\thispagestyle{empty}

\newpage

\pagenumbering{arabic} 

\section{Introduction}\label{intro}

\section{Literature Review}
\section{Methodology}
\section{Results}
\subsection{Model}
Suppose there are $N$ robots with compliance group $R_c = \{1,2,\dots,n_c\}$ and defiance group $R_d = \{n_{c+1},\dots,N\}$. Then, the dynamics is denoted as followed
\begin{align*}
  \frac{\partial}{\partial t} P_{\ell}(x,t) &= D_\ell \nabla^2 P_\ell (x,t) - \gamma_\ell P_\ell(x,t) + S_{\ell}(x,t)\\
  S_\ell(x,t) &= \sum_{k\in R_c} s_\ell\delta(x-X_t^{(k)}) + \sum_{m\in R_d}s_\ell\delta(x-Y_t^{(m)})\\
  dX_t^{(k)} &= Z_t^{(k)}\left[\mu(P_\ell(X_t^{(k)},t),t)dt + \sigma dW_t\right] + (1-Z_t^{(k)})\left[\sigma dW_t\right]\\
  Z_t^{(k)} &\sim \operatorname{Bernoulli}(p^{(k)}_t)\\
  Y_t^{(k)} &\sim \operatorname{Unknown, adversarial}
\end{align*}
\end{document}


