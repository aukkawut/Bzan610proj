\documentclass[12pt]{article}
\usepackage{booktabs}
\usepackage{amssymb,amsmath, amsthm, endnotes,setspace,enumerate,ulem,color,xspace,lscape}
\usepackage{amsthm,subfigure,graphics,epsfig, hyperref}
\usepackage[sort]{natbib}
\allowdisplaybreaks[1]
\normalem
\newcommand{\textR}[1]{\textcolor{blue}{\texttt{#1}}}
\newcommand{\R}{\textR{R}}
\newtheorem{post}{Postulate}
\newtheorem{theorem}{Theorem}
\newtheorem{definition}{Definition}
\begin{document}

\title{Adversarial Pheromone-based Swarm Robots Dynamics as System of Stochastic Partial Differential Equations: Continuous-time Information Design Perspective}  

\author{
Aukkawut Ammartayakun   \thanks{Bredesen Center for Interdisciplinary Research and Graduate Education, University of Tennessee, Knoxville.  \\Email: {\tt aammartayakun@tennessee.edu}}  
}

\maketitle

\begin{center}
\textbf{Abstract}
\end{center}

In the foraging task, the goal for agents are to reach the objective and to accumulate said objective by bringing those back to the start. However, as shown by Aswale et al. (2022), if the information is communicate through the pheromone, then there exists the adversarial strategy that can sabotage the task. In this work, the theoretical framework for the adversarial foraging task will be presented along with the insights from the information theory perspective. System of stochastic partial differential equations is used to formulate the framework along with mean-field theory. 

\noindent
\textbf{Keywords}: Byzantine Model, Foraging Task, Swarm Robotics, Adversarial Behavior, Pheromone-based Robots

\thispagestyle{empty}

\newpage

\pagenumbering{arabic} 

\section{Introduction}\label{intro}

\section{Literature Review}
\section{Results}
\subsection{Model}
Suppose there are $N$ robots with compliance group $R_c = \{1,2,\dots,n_c\}$ and defiance group $R_d = \{n_{c+1},\dots,N\}$. Then, the dynamics is denoted as followed
\begin{align}
\begin{split}
  \frac{\partial}{\partial t} P_{\ell}(x,t) &= D_\ell \nabla^2 P_\ell (x,t) - \gamma_\ell P_\ell(x,t) + S_{\ell}(x,t)\\
  S_\ell(x,t) &= \sum_{k\in R_c} s_\ell\delta(x-X_t^{(k)}) + \sum_{m\in R_d}s_\ell\delta(x-Y_t^{(m)})\\
  dX_t^{(k)} &= Z_t^{(k)}\left[\mu(P_\ell(X_t^{(k)},t),t)dt + \sigma dW_t\right] + (1-Z_t^{(k)})\left[\sigma dW_t\right]\\
  Z_t^{(k)} &\sim \operatorname{Bernoulli}(p^{(k)}_t)\\
  Y_t^{(k)} &\sim \operatorname{Unknown, adversarial}
  \end{split}
\end{align}
\subsection{Existence of Absorbing path}
Intuitively, one can think of one example of such a path as two scenario:
\begin{enumerate}
    \item Diverging path: in this scenario, the adversaries reinforcing the path that leads to nowhere (i.e., diverging) resulting in delay, if not, trap (if that diverging path leads to absorbing path) those agents in the loop
    \item Potential Wall: in this scenario, the adversaries reinforce the wall so much so that it create potential wall or barrier that ant cannot pass despite it is connecting to the target path.
\end{enumerate}
Mathematically, we can formulate the following scenario as:
\begin{equation}
  \exists \Gamma : \mathbb{R} \rightarrow \mathbb{R}^2 \text{ path}, \exists \tau \in \mathbb{R}^{+},\forall t > \tau \cup \{0\}, X_\tau^{(k)}\in \Gamma \implies \mathbb{P}(X_\tau^{(k)}\not\in \Gamma) = 0
\end{equation}
That is, there exits a path such that once the agent cross that path, almost surely they cannot leave that path. In this sense, if this were to be discrete space, it is equivalent to reaching the absorbing path. However, since the pheromone is decaying by time as per our dynamics, and also the agent has tendency to ignore pheromone and perform random walk, such the path is now non-trivial.

Now, given the dynamics, we can determine the existence of the path by solving for the evolution of the spatial distribution of the pheromone field. In order to accomplish that, consider the simple case of two agents, one is compliance and one is defiance.

The dynamics are now simplified into
\begin{align}
\begin{split}
  \frac{\partial}{\partial t} P_{\ell}(x,t) &= D_\ell \nabla^2 P_\ell (x,t) - \gamma_\ell P_\ell(x,t) + s_\ell(\delta(x-X_t)+\delta(x-Y_t)\\
  dX_t&= Z_t\left[\mu(P_\ell(X_t,t),t)dt + \sigma dW_t\right] + (1-Z_t)\left[\sigma dW_t\right]\\
  Z_t &\sim \operatorname{Bernoulli}(p_t)\\
  Y_t^{(k)} &\sim \operatorname{Unknown, adversarial}
  \end{split}
\end{align}
\subsection{Dynamic Discovery}
\section{Discussion}
\section{Future Works}
\section{Reference}
\end{document}


