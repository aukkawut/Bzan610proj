\documentclass[10pt]{beamer}
\usepackage{tikz}
\usepackage{pgfplots}
\usepackage{mathtools}
\usepackage{physics}
\usepackage{amsfonts,amsmath,amssymb,amsthm}
\usetikzlibrary{shapes.geometric, arrows}

\tikzstyle{startstop} = [rectangle, rounded corners, minimum width=3cm, minimum height=1cm, text centered, draw=black, fill=red!30]
\tikzstyle{process} = [rectangle, minimum width=3cm, minimum height=1cm, text centered, draw=black, fill=blue!30]
\tikzstyle{decision} = [diamond, minimum width=3cm, minimum height=1cm, text centered, draw=black, fill=green!30]
\tikzstyle{arrow} = [thick,->,>=stealth]

\usetheme[progressbar=frametitle]{metropolis}
\usepackage{appendixnumberbeamer}

\usepackage{booktabs}
\usepackage[scale=2]{ccicons}

\usepackage{pgfplots}
\usepgfplotslibrary{dateplot}

\usepackage{xspace}
\newcommand{\themename}{\textbf{\textsc{metropolis}}\xspace}

\title{Information Structure and Influence Maximization Case Studies}
%\subtitle{Case study: Information Operation}
% \date{\today}
\date{}
\author{Aukkawut Ammartayakun}
\institute{University of Tennessee, Knoxville}
% \titlegraphic{\hfill\includegraphics[height=1.5cm]{logo.pdf}}

\definecolor{utorange}{RGB}{255, 130, 0}
\setbeamercolor{frametitle}{bg=utorange, fg=white}
\setbeamercolor{background canvas}{bg=white}
\begin{document}

\maketitle
\begin{frame}{Table of contents}
  \setbeamertemplate{section in toc}[sections numbered]
  \tableofcontents%[hideallsubsections]
\end{frame}
\section{Social Network Influence Maximization}
\begin{frame}{What's Wrong With Previous Works?}
    \begin{itemize}
        \item Information cascade is modeled as change in behavior. Not how the information is propagated.
        \item Are one step: once you adopted the behavior, you are done.
    \end{itemize}
\end{frame}
\begin{frame}{Information Cascade in Graph}
\begin{figure}
    \centering
    \includegraphics[width=1\linewidth]{graph1.pdf}
    \caption{\textbf{Information Cascade on Directional Social Network}. Individual node has their own belief level $\alpha \in [-1,1]$.}
    \label{fig:1}
\end{figure}
\end{frame}
\section{Spatial-Temporal Influence Maximization in Swarm Robotics}
\begin{frame}{Pheromone-Based Robot Foraging}
\begin{figure}
    \centering
    \includegraphics[width=1\linewidth]{DetractorWall.png}
    \caption{\textbf{Effect of Malicious Actor on the Foraging.} Green is food, blue circle is nest. Blue trails is to-food pheromone and red is adversarial to-food pheromone.}
    \label{fig:2}
\end{figure}
\end{frame}
\begin{frame}{Pheromone-Based Robot Foraging: Cautionary Pheromone}
\begin{figure}
    \centering
    \includegraphics[width=0.8\linewidth]{counter_timeline.png}
    \caption{\textbf{Time Evolution Of Pheromone Concentration Given Cautionary Pheromone} (Aswale et al., AAMAS 2022)}
    \label{fig:2}
\end{figure}
\end{frame}
\begin{frame}{Pheromone and Robot Dynamics}
\begin{itemize}
    \item Suppose $i=1,\dots,N$ are robots that locate at $x_i(t)$. At position $x(t)$, there are 2 pheromones concentration $P_\text{food},P_{\text{home}}$
    \[
    \dd x_i(t) = \overbrace{v_i(t) \dd t}^\text{correlated term} + \underbrace{\sigma \dd W_i(t)}_{\text{random walk}}
    \]
    \[
    \dfrac{\partial}{\partial t} P_\ell(x,t) = \overbrace{D_\ell \nabla^2P_\ell(x,t)}^{\text{diffusion}} - \underbrace{\lambda_\ell P_\ell(x,t)}_{\text{evaporation}} +\overbrace{S_\ell(x,t)}^{\text{deposition}}, \quad \ell\in\{\text{food},\text{home}\}
    \] (Ryan, Journal of Mathematical Biology, 2016)
    \item Note that pheromone dynamics is SPDE as \[S_\ell(x,t) = \sum_{i=1}^N s_i(t)\delta(x-X_i(t))\] for random position $X_i(t)$ described by dynamics $\dd x_i(t)$
    \item Robot has prior belief in trustworthiness of the observation.
\end{itemize}
\end{frame}
\section{Methodology}
\begin{frame}{Method}
    \begin{itemize}
        \item Monte Carlo simulation of different settings as initial dynamics.
        \begin{itemize}
            \item In discrete case, we can look at the Erd\H{o}s-R\'enyi social network with different families of distributions of propagation: uniform, heavy-tailed, Gaussian, etc. Or using the synchronization/quorum sensing as the way to model the dynamics.
            \item Continuous case, we have some prior data and simulator from previous work.
        \end{itemize}
        \item Theoretical exploration
        \begin{itemize}
            \item Both are evolution of distribution which is the solution of SDE (and also PDE). So, Kolmogorov forward equation or Fokker-Planck equation might needed to be considered.
        \end{itemize}
        \item Information Structure
        \begin{itemize}
            \item How would knowing more information (say if robot can communicate, or we know which side that person is on) changes the strategy?
        \end{itemize}
    \end{itemize}
\end{frame}

\end{document}
            